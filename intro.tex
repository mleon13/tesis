\section{\textbf{Introducci\'on}}\label{introduccion}
El Instituto del Caf\'e de Costa Rica (ICAFE) es una instituci\'on p\'ublica de car\'acter no estatal, fundada en 1933 como rectora de la caficultura costarricense. Uno de los objetivos principales de ICAFE es investigar y desarrollar tecnolog\'ia agr\'icola e industrial. Guiados por este objetivo, ICAFE busca hacer uso de la tecnolog\'ia para resolver problemas y mejorar procesos en el \'area de la producci\'on agr\'icola. Uno de estos problemas es el conteo de granos de caf\'e en im\'agenes. Para llevar a cabo esta tarea se requiere tiempo y horas hombre. Cuando los ojos trabajan por mucho tiempo, causa fatiga y eso a su vez da paso a posibles errores en el conteo de granos. Se desea hacer uso de la visi\'on por computadora para encontrar una posible soluci\'on a este problema.
\\\\
El presente trabajo trata sobre la detecci\'on de granos de caf\'e existente en im\'agenes.
Se estudia el estado del arte en el \'area de detecci\'on de objetos en im\'agenes,
espec\'icamente en granos. Posteriormente se propone la incorporaci\'on del algoritmo
\textit{"Fast Radial Symmetry Transform"}\cite{loyzelinsky} en la herramienta de c\'odigo abierto P-TRAP\cite{ptrap} para incrementar la exactitud (accuracy) en la detecci\'on de granos de caf\'e existentes.
\\\\
Para poder desarrollar este trabajo se estudiar\'a a fondo los conceptos del algoritmo \textit{"Fast Radial Symmetry Transform"}\cite{loyzelinsky} para poder ser incoporados a la herramienta P-TRAP\cite{ptrap}.
Luego, se obtendr\'an una serie de im\'agenes de granos de caf\'e por parte de ICAFE. Despu\'es se prepar\'a el ambiente de desarrollo de la herramienta P-TRAP\cite{ptrap} con el fin de realizar las moficaciones pertinentes para incorporar el algoritmo \textit{"Fast Radial Symmetry Transform"}\cite{loyzelinsky} al proceso de detecci\'on de granos existente. Finalmente, se ejecutar\'an una serie de experimentos sobre la base de datos de im\'agenes de granos de caf\'e entre la versi\'on desarrollada de P-TRAP\cite{ptrap} y su versi\'on por defecto con el prop\'osito de medir la exactitud (accuracy) en la detecci\'on de granos de caf\'e existentes en una imagen.
\\\\
El principal resultado de esta tesis es una versi\'on de P-TRAP\cite{ptrap} que incorpore el algoritmo \textit{"Fast Radial Symmetry Transform"}\cite{loyzelinsky}, permitiendo una detecci\'on m\'as precisa de granos de caf\'e existentes en una imagen en comparaci\'on a P-TRAP\cite{ptrap} en su versi\'on por defecto. 
\\\\
En la secci\'on \ref{marcoteorico} se presenta el marco te\'orico. Posteriormente, en la secci\'on \ref{propuesta}, se expone la propuesta de proyecto. Seguidamente, los objetivos generales y espec\'ificos son presentados en la secci\'on \ref{objetivos}. El alcance y limitacaciones son parte de la secci\'on \ref{alcance}. En la secci\'on \ref{metodologia} se presenta la metodolog\'ia que se va utilizar en este proyecto de investigaci\'on. Finalmente, en la secci\'on \ref{plan} se incluye el plan de trabajo que va ser utilizado durante esta investigaci\'on.



