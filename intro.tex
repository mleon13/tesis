\section{\textbf{Introducci\'on}}\label{introduccion}
El presente trabajo trata sobre la detecci\'on de granos de caf\'e existente en im\'agenes. 
Se procede a estudiar el estado del arte en el \'area de detecci\'on de objetos en im\'agenes, espec\'ificamente en granos. 
Posteriormente se propone la incorporaci\'on del algoritmo \textit{"Fast Radial Symmetry Transform"}\cite{loyzelinsky} en la herramienta de c\'odigo abierto P-TRAP\cite{ptrap} para incrementar la precisi\'on en la detecci\'on de granos de caf\'e existentes. 
\\\\
Para poder desarrollar este trabajo se obtuvo una serie de im\'agenes de granos de caf\'e por parte del Instituto del Caf\'e de Costa Rica (ICAFE). Se prepar\'o el ambiente de desarrollo de la herramienta P-TRAP\cite{ptrap} con el fin de realizar las moficaciones pertinentes para incorporar el algoritmo \textit{"Fast Radial Symmetry Transform"}\cite{loyzelinsky} al proceso de detecci\'on de granos existente. Se ejecutan una serie de experimentos sobre la base de datos de im\'agenes de granos de caf\'e entre la versi\'on desarrollada de P-TRAP\cite{ptrap} y su versi\'on por defecto con el prop\'osito de medir la precisi\'on en la detecci\'on de granos de caf\'e existentes en una imagen.
\\\\
El principal resultado de esta tesis es una versi\'on de P-TRAP\cite{ptrap} que incorpore el algoritmo \textit{"Fast Radial Symmetry Transform"}\cite{loyzelinsky}, permitiendo una detecci\'on m\'as precisa de granos de caf\'e existentes en una imagen en comparaci\'on a P-TRAP\cite{ptrap} en su versi\'on por defecto. 
\\\\
En la secci\'on \ref{marcoteorico} se presenta el marco te\'orico. Posteriormente, en la secci\'on \ref{propuesta}, se expone la propuesta de proyecto. Seguidamente, los objetivos generales y espec\'ificos son presentados en la secci\'on \ref{objetivos}. El alcance y limitacaciones son parte de la seccion \ref{alcance}. En la secci\'on \ref{metodologia} se presenta la metodolog\'ia que se va utilizar en este proyecto de investigaci\'on. Finalmente, en la secci\'on \ref{plan} se incluye el plan de trabajo que va ser utilizado durante esta investigaci\'o.



