\section{\textbf{Metodolog\'ia}}\label{metodologia}
\subsection{Dise\~no de Experimentos}
Para describir el planeamiento pre-experimental para el dise\~no de experimentos de este trabajo, (con la informaci\'on disponible hasta el momento), se usan los \textit{lineamientos} desarrollados en el libro de Douglas C. Montgomery \cite{montgomeryx}. El esquema del procedimiento recomendado en los lineamientos para esta etapa incluye lo siguiente:
\begin{itemize}
\item [1.] \textbf{Reconocimiento y definici\'on del problema:} consiste en desarrollar una declaraci\'on clara y sencilla del problema. Una clara definici\'on del problema, normalmente contribuye substancialmente a una mejor comprensi\'on del fen\'omeno que est\'a siendo estudiado y a la soluci\'on final de dicho problema.
\item [2.] \textbf{Selecci\'on de factores, niveles y rangos:} consiste en enumerar todos los posibles factores que pueden influenciar el experimento. Incluye tanto los factores de dise\~no potencial (los que potencialmente se podr\'ian querer modificar en los experimentos) y los factores perturbadores (los que no se quieren estudiar en el contexto del experimento). Tambi\'en se deben seleccionar los rangos sobre los que var\'ian los distintos factores y los niveles espec\'ificos sobre los que se aplicar\'an las iteraciones del experimento.
\item [3.] \textbf{Selecci\'on de la variable de respuesta:} debe proveer informaci\'on \'util sobre el fen\'omeno que esta siendo estudiado.
\item [4.] \textbf{Selecci\'on del dise\~no de experimental:} se refiere a aspectos claves del experimento tales como el tama\~no de la muestra, la selecci\'on del orden adecuado para la ejecuci\'on de los intentos experimentales y la decisi\'on de bloquear o no algunas de las restriciones de aleatoriedad en la pruebas.
\item [5.] \textbf{Llevar a cabo el experimiento:} en esta etapa, es de vital importancia monitorear el proceso cuidadosamente para asegurar la correcta ejecuci\'on del experimento con respecto a lo planeado.
\end{itemize}
\subsubsection{Declaraci\'on del problema}
Estudiar la precisi\'on en la detecci\'on de granos de caf\'e existentes por parte de la versi\'on por defecto de P-TRAP\cite{ptrap} y la versi\'on desarrollada en este estudio incorporando el algoritmo \textit{Fast Radial Symmetry Transform}\cite{loyzelinsky}.
\subsubsection{Factores}
En el dise\~no de experimentos, un factor es aquel componente que tiene cierta
influencia en las variables de respuesta \cite{montgomeryx}. El objetivo de un experimento es determinar esta influencia.
\\\\
Usando la informaci\'on recolectada en esta etapa de la investigaci\'on, as\'i como la experiencia adquirida por el estudiante y expuesta en las secciones anteriores, se han seleccionado inicialmente los siguientes factores para su estudio:
\begin{enumerate}
    \item Resoluci\'on de la imagen
	    \begin{itemize}
		    \item Baja (Rango por definir)
		    \item Media (Rango por definir)
		    \item Alta (Rango por definir)
	    \end{itemize}
    \item Formato de la imagen
	    \begin{itemize}
	        \item BMP (Bitmap)
	        \item GIF
	        \item JPG-JPEG
	        \item PNG
	    \end{itemize}
\end{enumerate}
Potencialmente, la cantidad de factores que influyen en las variables de respuesta es infinita; sin embargo, por razones de tiempo y presupuesto se han seleccionado las que se considera ejercen mayor influencia en las variables de respuesta. Algunos factores que no se considerar\'an en esta investigaci\'on
pero que tambi\'en pueden ejercer influencia son:
\begin{itemize}
\item Pre-procesamiento aplicado a muestras
\item Otros par\'ametros muestrales
\end{itemize}
\subsubsection{Variables de respuesta}
La variable de respuesta corresponde a la m\'etrica detallada en la secci\'on \ref{metricas} de este documento:
\begin{itemize}
\item Precisi\'on
\end{itemize}
\subsubsection{Recolecci\'on de datos}
Una vez ejecutados los experimentos durante el desarrollo de la tesis, se deben recolectar los resultados de estos. La obtenci\'on de las distintas variables de respuesta la har\'a de forma autom\'atica la plataforma que se desarrollar\'a. 
\subsubsection{An\'alisis estad\'istico}
El an\'alisis de varianza unifactorial por rangos, de Kruskal-Wallis\cite{kruskalwallis}, es una prueba \'util para decidir si \textit{k} muestras independientes provienen de diferentes poblaciones.
\\\\
La t\'ecnica de Kruskal-Wallis\cite{kruskalwallis} prueba la hip\'otesis nula de que las \textit{k} muestras provienen de la misma poblaci\'on o de poblaciones id\'enticas con la misma mediana. Para especificar expl\'icitamente la hip\'otesis nula y alterna, $\theta_j$ debe ser la mediana de la poblaci\'on para el j-\'esimo grupo o muestra.
Entonces podemos escribir la hip\'otesis nula de que las medianas son las mismas como:
\[H_0 : \theta_0 = \theta_1 = ...\theta_k \] 
Por su parte la hip\'otesis alterna puede ser escrita de la siguiente forma (para algunos grupos \textit{i} y \textit{j}):
\[H_1 : \theta_i \neq \theta_j \] 
Si la hip\'otesis alterna es verdadera, al menos un par de grupos tienen medianas diferentes. Seg\'un la hip\'otesis nula, la prueba supone de las variables en estudio tienen la misma distribuci\'on subyacente \cite{monicamontano}.
\subsection{Ambiente de desarrollo}
El desarrollo de este proyecto se va realizar en una m\'aquina de 64 bits con Windows 8.1 Enterprise. La herramienta P-TRAP\cite{ptrap} fue desarrollada en el lenguaje de programaci\'on Java por lo que nosotros utilizaremos NetBeans \cite{netbeans} en su versi\'on 8.0.1 como nuestro ambiente integrado de desarrollo. Asimismo, utilizaremos la versi\'on 9.0 de MATLAB \cite{matlab} para la integraci\'on con el algoritmo \textit{“Fast Radial Symmetry Transform”}\cite{loyzelinsky}.
