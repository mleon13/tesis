\section{\textbf{Propuesta de Proyecto}}
\subsection{Planteamiento del Problema}
La visi\'on por computadora es una disciplina cient\'ifica que utiliza distintos m\'etodos para adquirir, procesar, analizar y comprender im\'agenes del mundo real. 
\\\\
Durante la \'ultima d\'ecada hemos sido testigos del impresionante avance tecnol\'ogico de las c\'amaras fotogr\'aficas. Actualmente, vivimos en una era donde contamos con dispositivos m\'oviles capaces de tomar fotograf\'ias de alta calidad en cualquier lugar y momento. 
\\\\
Este avance ha sido el detonante para que en a\~nos recientes la visi\'on por computadora se haya convertido en una tecnolog\'ia clave en diferentes campos. En la industria automotriz son cada vez m\'as populares los carros asistidos por computadoras. Tambi\'en, en la industria de videojuegos la visi\'on por computadora ha sido utilizada para mejorar la interacci\'on entre el usuario y el juego, creando una experiencia de juego cada vez m\'as real. En el \'area de la producci\'on agr\'icola, la visi\'on por computadora puede ser utilizada para mejorar la eficiencia y eficacia en los procesos de selecci\'on y conteo de granos. Ambos procesos, de ser realizados de forma manual, consumen mucho tiempo y existe cierto grado de error humano que afecta el resultado final.
\\\\
En el Instituto del Caf\'e de Costa Rica (ICAFE) se cuenta con algoritmos para estimar la producci\'on de una planta en particular. Sin embargo, dichos algoritmos requieren como par\'ametro el n\'umero de granos de caf\'e en la planta en cierto punto de su desarrollo. Actualmente, el proceso de conteo de granos se realiza de forma manual, consumiendo una gran cantidad de horas hombre. ICAFE cuenta con una serie de im\'agenes de granos de caf\'e y se desea encontrar una manera autom\'atica de realizar el conteo de granos con el fin de incrementar la eficiencia en el proceso.
\\\\\\
En la literatura, se han propuesto varias soluciones para atacar este problema. Sin embargo, la gran mayor\'ia utiliza como entrada im\'agenes de granos en un ambiente controlado. Este es el caso de P-TRAP, una herramienta de c\'odigo abierto que implementa un algoritmo para la detecci\'on de granos en un ambiente controlado. Sin embargo, ICAFE no cuenta con un ambiente controlado para la toma de im\'agenes y cada imagen es tomada directamente desde la planta, en su ambiente natural. Por lo que no se puede hacer uso de esta funci\'on en P-TRAP para realizar el conteo a partir de las imagenes de ICAFE.
\subsection{Propuesta del Proyecto}
A partir de lo descrito anteriormente, este proyecto pretende estudiar la eficacia en el conteo de granos a partir de la incorporaci\'on del algoritmo \textit{“Fast Radial Symmetry Transform”} al proceso de conteo de granos ya existente en la herramienta P-TRAP.
\subsection{Trabajos Relacionados}
\subsection{Hip\'otesis}
Con base en la definici\'on del problema y en la propuesta de proyecto, se define la siguiente hip\'otesis:\\\\
\textbf{\textit{La incorporaci\'on del algoritmo “Fast Radial Symmetry Transform” para la detecci\'on de granos de caf\'e en la herramienta de c\'odigo abierto P-TRAP aumenta la cantidad de detecciones de granos de caf\'e existentes en una imagen.}} 
\subsection{M\'etricas}
\subsection{Desarrollo del Proyecto}