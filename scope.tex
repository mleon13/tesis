\section{\textbf{Alcance y Limitaciones}}\label{alcance}
Dentro del alcance de esta propuesta se define el siguiente conjunto de productos a ser presentados durante la defensa final de los resultados del mismo:
\begin{itemize}
\item Una versi\'on de P-TRAP\cite{ptrap} que incluya el algoritmo \textit{Fast Radial Symmetry Transform}\cite{loyzelinsky} en el proceso de detecci\'on de granos.
\item Todas las funcionalidades de P-TRAP\cite{ptrap} estar\'an disponibles en la nueva versi\'on.
\item Un an\'alisis estad\'istico para contrastar los resultados de los experimentos.\\
\end{itemize}
Es necesario delimitar esta investigaci\'on por motivos de tiempo y extensi\'on. Se plantean entonces la siguientes limitaciones:
\begin{itemize}
\item Solo se contar\'an los granos de caf\'e visibles en la foto. 
\item Solo se realizar\'a el conteo de granos de caf\'e para im\'agenes RGB.
\item Se debe proporcionar la escala en la imagen.
\item Queda por fuera de este trabajo el conteo de granos traslapados 
\item No se contar\'an granos de caf\'e no visibles en la imagen. Es decir, aquellos granos de caf\'e que se encuentren detr\'as del r\'acimo en la foto.
\end{itemize}